\documentclass[a4paper,12pt]{article}
\usepackage[utf8]{inputenc}
\usepackage[ngerman]{babel}
\usepackage{graphicx}
\usepackage{hyperref}
\usepackage{geometry}
\geometry{a4paper, margin=2.5cm}

% Titelinformationen
\title{Werkzeuge für das wissenschaftliche Arbeiten}
\author{}
\date{Python for Machine Learning and Data Science \\ Abgabe: 15.12.2023}

\begin{document}

% Titelseite
\maketitle

% Inhaltsverzeichnis
\tableofcontents
\newpage

% 1. Projektaufgabe
\section{Projektaufgabe}

In dieser Aufgabe beschäftigen wir uns mit Objektorientierung in Python. Der Fokus liegt auf der Implementierung einer Klasse, dabei nutzen wir insbesondere auch Magic Methods.

\begin{figure}[h!]
    \centering
    \includegraphics[width=0.8\textwidth]{image.pdf} % Hochqualitative Grafik
    \caption{Darstellung der Klassenbeziehungen.}
    \label{fig:diagram}
\end{figure}

\subsection{Einleitung}

Ein Datensatz besteht aus mehreren Daten. Ein einzelnes Datum wird durch ein Objekt der Klasse \texttt{DataSetItem} repräsentiert. Jedes Datum hat:
\begin{itemize}
    \item Einen Namen (Zeichenkette),
    \item Eine ID (Zahl),
    \item Beliebigen Inhalt.
\end{itemize}

Mehrere Daten (Objekte vom Typ \texttt{DataSetItem}) werden in einem Datensatz zusammengefasst. Die Klasse \texttt{DataSetInterface} definiert die Schnittstelle und die unterstützten Operationen eines Datensatzes.

\subsection{Aufbau}

Es gibt drei Dateien:
\begin{itemize}
    \item \texttt{dataset.py}: Enthält die Klassen \texttt{DataSetInterface} und \texttt{DataSetItem}.
    \item \texttt{implementation.py}: Hier wird die Klasse \texttt{DataSet} implementiert.
    \item \texttt{main.py}: Nutzt die Klassen \texttt{DataSet} und \texttt{DataSetItem} und testet deren Funktionalität.
\end{itemize}

\subsection{Methoden}

Folgende Methoden der Klasse \texttt{DataSet} sind zu implementieren:
\begin{itemize}
    \item \texttt{\_\_setitem\_\_(self, name, id\_content)}: Hinzufügen eines Datums mit Name, ID und Inhalt.
    \item \texttt{\_\_iadd\_\_(self, item)}: Hinzufügen eines \texttt{DataSetItem}.
    \item \texttt{\_\_delitem\_\_(self, name)}: Löschen eines Datums anhand des Namens.
    \item \texttt{\_\_contains\_\_(self, name)}: Prüfen, ob ein Datum mit diesem Namen vorhanden ist.
    \item \texttt{\_\_getitem\_\_(self, name)}: Abrufen eines Datums über seinen Namen.
    \item \texttt{\_\_and\_\_(self, dataset)}: Schnittmenge zweier Datensätze bestimmen und als neuen Datensatz zurückgeben.
    \item \texttt{\_\_or\_\_(self, dataset)}: Vereinigungen zweier Datensätze bestimmen und als neuen Datensatz zurückgeben.
    \item \texttt{\_\_iter\_\_(self)}: Iteration über alle Daten des Datensatzes.
    \item \texttt{filtered\_iterate(self, filter)}: Gefilterte Iteration mit einer Lambda-Funktion.
    \item \texttt{\_\_len\_\_(self)}: Anzahl der Daten abrufen.
\end{itemize}

% 2. Abgabe
\section{Abgabe}

Programmieren Sie die Klasse \texttt{DataSet} in der Datei \texttt{implementation.py}, um die oben beschriebenen Anforderungen zu erfüllen. Nutzen Sie die bereitgestellten Dateien aus dem Moodle.

Das VPL prüft den Code mithilfe erweiterter Testfälle in der \texttt{main.py}. Alle Testfälle sind erfolgreich, wenn \texttt{CodeIsValid} ausgegeben wird.

\end{document}